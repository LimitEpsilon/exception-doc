\documentclass{article}
\usepackage{geometry}
\geometry{
 a4paper,
 total={170mm,257mm},
 left=20mm,
 top=20mm,
 }
\usepackage[utf8]{inputenc}
\usepackage{graphicx}
\usepackage{kotex}
\usepackage{amsmath}
\usepackage{amsthm}
\usepackage{amssymb}
\usepackage{ebproof}
\usepackage{fancyvrb}
\usepackage{biblatex}
\usepackage{csquotes}
\newtheorem{clm}{Claim}[section]
\addbibresource{citations.bib}

% fonts
\usepackage[math-style=TeX,bold-style=TeX]{unicode-math}

\usepackage{libertinus}

\newfontfamily{\fallbackfont}{EB Garamond}
\DeclareTextFontCommand{\textfallback}{\fallbackfont}
\usepackage{newunicodechar}
\newunicodechar{⩴}{\textfallback{⩴}}

\setmainhangulfont{Noto Serif CJK KR}
\setsanshangulfont[BoldFont={* Bold}]{KoPubWorld Dotum.ttf}
\setmonohangulfont{D2Coding}

\usepackage{simplebnf}

% because of simplebnf
\newcommand\vbar{|}
\newcommand*{\finto}{\xrightarrow{\text{\textrm{fin}}}}
\newcommand*{\istype}{\mathrel{⩴}}
\newcommand*{\ortype}{\mathrel{|}}

\title{Constructing Set Constraints for ReScript}
\author{서울대학교 전기$\cdot$정보공학부 2018-12602 이준협}
\date{}
\begin{document}
\maketitle
\section{Definition of set expressions}
\begin{bnfgrammar}
  $se$ ::= $\varnothing$ : empty set
  | $\_$ : maximum set
  | $()$ : unit
  | $n$ : integer
  | $b$ : boolean
  | $\lambda x.e$ : function
  | $V_{e}$ : set variable corresponding to the possible values of $e$
  | $P_{e}$ : set variable corresponding to the possible exn packets of $e$
  | $\mathsf{body}_{V}(se)$ : values that $se$ can spit out when $se$ is applied to something
  | $\mathsf{body}_{P}(se)$ : exn packets that $se$ can spit out when $se$ is applied to something
  | $\mathsf{par}(se)$ : values that can be a parameter for $se$
  | $\kappa$ : constructor
  | $l$ : field of a record
  | $\mathsf{con}(\kappa,se)$ : construct
  | $\mathsf{exn}(\kappa,se)$ : exception
  | $\mathsf{fld}(se, l)$ : contents of the field $l$ of a record $se$
  | $\mathsf{cnt}(se)$ : contents of a reference
  | $\mathsf{bop}(se, se)$ : binary operators, where $\mathsf{bop}\in\{+, -, \times, \div, =, <, >\}$
  | $f^{-1}_{(i)}(se)$ : projection onto the i-th argument of $f$
  | $se \cup se$ : union
  | $se \cap se$ : intersection
\end{bnfgrammar}

The definition of the conditional set expression needs clarification.

\[
  se_{1} \Rightarrow se_{2} \coloneq
  \begin{cases}
    \varnothing & (se_{1}=\varnothing)\\
    se_{2} & (o.w)
  \end{cases}
\]

The conditional set expression is a naive approximation for pattern matching. Consider the case when we want to match an expression against the record pattern with fields $x$ and $y$.
The constraints describing the record $r = \{x = 1, y = 2\}$ are $1 \subseteq \mathsf{fld}(V_{r}, x) \wedge 2 \subseteq \mathsf{fld}(V_{r}, y)$.
To pattern-match $r$ against $\{x, y\}$, we want the fields $x$ and $y$ of $V_{r}$ to be nonempty.
Thus, the value of ``$match\:r\:with\:\{x, y\}\rightarrow e$'' is $\mathsf{fld}(V_{r}, x)\Rightarrow(\mathsf{fld}(V_{r},y)\Rightarrow V_{e})$.

The conditional set expression can also be used to define conditional set constraints \cite{Aik99}.
\[
  se \Rightarrow \bigwedge_{i=1}^{n}{X_{i}\subseteq Y_{i}} \coloneq \bigwedge_{i=1}^{n}{(se \Rightarrow X_{i}) \subseteq Y_{i}}
\]

\section{Constructing set constraints}

Now we are in a position to define constraint construction rules for our ReScript-like language. Hopefully this would be reasonably fast when implemented and be accurate enough...

\begin{center}
\begin{prooftree}
    \infer[left label={UNIT, INT, BOOL}]0[$c=(),n,b$] {\rhd c \colon V_{e}\supseteq c}
\end{prooftree}

\begin{prooftree}
  \hypo{\rhd e_{1} \colon C_{1}}
  \hypo{\rhd e_{2} \colon C_{2}}
  \infer[left label={APP}]2{\rhd e_{1} \: e_{2} \colon (V_{e}\supseteq \mathsf{body}_{V}(V_{e_{1}}))\wedge
    (P_{e}\supseteq (\mathsf{body}_{P}(P_{e_{1}}) \cup P_{e_{1}} \cup P_{e_{2}})) \wedge
    (\mathsf{par}(V_{e_{1}}) \supseteq V_{e_{2}})\wedge
    C_{1}\wedge
    C_{2}}
\end{prooftree}

\begin{prooftree}
  \hypo{\rhd e' \colon C'}
  \infer[left label={FN}]1{\rhd \lambda x.e' \colon (V_{e}\supseteq \lambda x.e')\wedge
    (\mathsf{body}_{V}(V_{e})\supseteq V_{e'})\wedge
    (\mathsf{body}_{P}(V_{e})\supseteq P_{e'})\wedge
    (\mathsf{par}(V_{e})\subseteq V_{x})\wedge
    C'}
\end{prooftree}

\begin{prooftree}
  \hypo{\rhd e_1 \colon C_1}
  \hypo{\rhd e_2 \colon C_2}
  \infer[left label={LET}]2{\rhd let \: x=e_1 \: in \: e_2 \colon
    (V_x \supseteq V_{e_1})\wedge
    (P_x \supseteq P_{e_2})\wedge
    }
\end{prooftree}
\end{center}

\printbibliography
\end{document}
